%-----------------------------------------------------------
%   Capítulo 1 - Introdução
%-----------------------------------------------------------


\chapter{Introdução}

As medições dos eixos de controlo é um parâmetro muito importante para a robótica, pois é necessário saber a posição do equipamento em três dimensões, tradicionalmente é utilizado um acelerômetro para se obter o valor do \textit{roll}, \textit{pitch} e \textit{yaw}.
\\ No âmbito deste projeto a medição dos eixos de controlo vai ser realizada a partir da comparação da luminosidade entre vários sensores de luz.

\section{Enquadramento e Motivação}
O projeto surgiu no âmbito da cadeira de Projeto de sistemas embebidos (Prose), cadeira integrante do plano de estudos do Mestrado de Sistemas Autónomos, a proposta apresentada é o desenvolvimento de um sistema de que detecta a inclinação a partir da luz, a base são vários anéis que fazem a medição da intensidade da luz, sendo que enviam esses dados para um \textit{master} que processa os dados.

\section{Cenários de aplicação}
A aplicação é para a deteção da inclinação de um tubo no ambiente marinho.

